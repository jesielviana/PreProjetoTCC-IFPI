\documentclass[a4paper,12pt]{article}


%\usepackage[utf8]{inputenc}			
\usepackage[english,brazil]{babel}
\usepackage{framed}
\usepackage{hyperref}
\usepackage{amsmath}
\usepackage{graphicx}
\usepackage[colorinlistoftodos]{todonotes}
\usepackage{wrapfig}
\usepackage{lipsum}
\usepackage{listings}
\usepackage{color}
\usepackage{indentfirst}
\usepackage{times}
\usepackage{textcomp}
\usepackage{enumerate}
\usepackage{enumitem}

\usepackage[nottoc]{tocbibind} %adiciona referência ao índice
\usepackage{pgfgantt}

\newif\ifblackandwhite
\blackandwhitetrue

\usepackage{fontspec}%
\defaultfontfeatures{Ligatures=TeX}%
\setmainfont[%
   Numbers        = OldStyle ,
   ItalicFont     = LinLibertineOI ,
   BoldItalicFont = LinLibertineOBI ,
   BoldFont       = LinLibertineOB ,
]{LinLibertineO}%


\usepackage[hmargin=2cm,vmargin=2.5cm]{geometry}
\usepackage{etoolbox}
\usepackage{longtable}%
\AtBeginEnvironment{longtable}{%
  \addfontfeature{RawFeature=+tnum;-onum}%  <--- requires LuaTeX
}

\usepackage{pdflscape}
%\usepackage[svgnames]{xcolor}
 \usepackage{colortbl}%
   \newcommand{\myrowcolour}{\rowcolor[gray]{0.925}}
\usepackage{booktabs}

\ifblackandwhite
  \newcommand{\cheading}[2]{\textbf{#1\hfill #2}}
  \newcommand{\highest}[1]{\textbf{#1}}% == highest score for question
\else
  \newcommand{\cheading}[2]{\textcolor{Maroon}{\textbf{#1\hfill #2}}}
  \newcommand{\highest}[1]{\textcolor{Maroon}{\textbf{#1}}}%
\fi

\definecolor{mygray}{rgb}{0.4,0.4,0.4}
\definecolor{mygreen}{rgb}{0,0.8,0.6}
\definecolor{myorange}{rgb}{1.0,0.4,0}

\lstdefinestyle{customc}{
  belowcaptionskip=1\baselineskip,
  breaklines=true,
  frame=L,
  xleftmargin=\parindent,
  language=C,
  showstringspaces=false,
  basicstyle=\footnotesize\ttfamily,
  keywordstyle=\bfseries\color{green!40!black},
  commentstyle=\itshape\color{purple!40!black},
  identifierstyle=\color{blue},
  stringstyle=\color{orange},
  numbers=left,
  numbersep=12pt,
  numberstyle=\small\color{mygray},
}
\lstset{escapechar=@,style=customc}

\newcommand{\HRule}{\rule{\linewidth}{0.5mm}}

\begin{document}

\begin{titlepage}
\begin{center}

% logo
\includegraphics[width=0.12\textwidth]{img/logo_ifpi.png}~\\[1cm]

\textsc{\Large Instituto Federal de Educação, Ciência e Tecnologia do Piauí\\[5cm]}



{\large \bfseries Título do Projeto de Monografia \\[3cm]}


% identificacao do proponente
\large\textbf{Proponente:}
<nome\_do\_proponente>\\[3cm]

\vfill

% Bottom of the page
{\large \today}

\end{center}
\end{titlepage}

\newpage
\tableofcontents

\newpage
\section{Teste}
\section{Introdução}

\label{sc:introducao}
Nesta seção o aluno deve apresentar o tema e contexto no qual está inserido seu projeto de TCC.


\section{Estado da Arte}
O Estado da Arte apresenta as principais referências(atuais) relacionadas ao tema
\ref{sc:introducao}

\section{Problema}
\label{sc:problema}
O problema que fomenta a trabalho deve ser apresentado no formato de uma pergunta


\section{Justificativa}
Nesta seção o aluno deve apresentar o que justifica o estudo a ser elaborado, ou seja o porquê que o tema deve ser estudado.

\section{Objetivos}
\label{sc:objetivos}
Nesta seção o objetivo geral e os objetivos específicos devem ser relacionados em itens por meio de verbos no infinitivo. Os objetivos devem ser simples e mensuráveis.

\subsection{Objetivo Geral}
Objetivo geral: indica o que se pretende alcançar de forma ampla e está relacionado à questão principal da pesquisa.

\subsection{Objetivos Específicos}
Objetivos específicos: contribuem para alcançar o objetivo geral, apontando as etapas que levam a consecução do objetivo maior. Observe a melhor sequência lógica, estabelecendo quais assuntos precedem a outros

\section{Metodologia}
\label{sc:metodologia}
Nesta seção deve ser descrita de forma detalhada a metodologia a ser empregada, ou seja  o caminho a ser percorrido para responder ao problema estabelecido. Deve estar de acordo com os
objetivos geral e específicos, abrangendo a definição de como será feito o trabalho.


%\section{Cronograma}
%\label{sc:cronograma}

%\newpage
\section{Orçamento detalhado}
\label{sc:orcamento}

\begin{longtable}{@{}l l l l}
% pairs: absolute number (percentage)

\toprule%
 \centering%
   \textbf{Item}
 & \textbf{Preço}
 & \textbf{Qtd.}
 & \textbf{Total (R\$)} \\


\cmidrule[0.4pt](r{0.125em}){1-1}%
\cmidrule[0.4pt](lr{0.125em}){2-2}%
\cmidrule[0.4pt](lr{0.125em}){3-3}%
\cmidrule[0.4pt](lr{0.125em}){4-4}%
% \midrule
\endhead


Item 1 & R\$ 1,00 & 1 & R\$ 1,00 \\
\myrowcolour%
Item 2 & R\$ 1,00 & 1 & R\$ 1,00 \\
Item 3 & R\$ 1,00 & 1 & R\$ 1,00 \\
\myrowcolour%
Item 4 & R\$ 1,00 & 1 & R\$ 1,00 \\
Item 5 & R\$ 1,00 & 1 & R\$ 1,00 \\
\myrowcolour%
Item 6 & R\$ 1,00 & 1 & R\$ 1,00 \\
Item 7 & R\$ 1,00 & 1 & R\$ 1,00 \\
\myrowcolour%
Item 8 & R\$ 1,00 & 1 & R\$ 1,00 \\
Item 9 & R\$ 1,00 & 1 & R\$ 1,00 \\

\bottomrule

\end{longtable}

\newpage
\section{Cronograma}
\label{sc:cronograma}

\subsection{Descrição das fases e metas do projeto}

Deve-se descrever as atividades a serem desenvolvidas e os marcos indicativos(componentes, equipamentos, textos, resultados de pequisas, software,etc.) que permitirão perceber o progresso das atividades.

\textbf{Etapa1: Estudo bibliográfico}

Atividades a serem desenvolvidas:
\begin{enumerate}[label=\textbf{1.\arabic*}]
\item \label{b1} Buscar por informações em diversos lugares
\item \label{b2}  Ler bastante 
\item \label{b3} Ler ainda mais
\end{enumerate}

\textbf{Etapa2: Desenvolvimento da parte inicial}

Atividades a serem desenvolvidas:
\begin{enumerate} [label=\textbf{2.\arabic*}]
\item \label{dev1} Buscar por ferramentas para o desenvolvimento
\item \label{dev2} Concepção do protótipo
\item \label{dev3} Testes e revisão do projeto inicial
\end{enumerate}

\textbf{Etapa3: Desenvolvimento da parte final}

Atividades a serem desenvolvidas:
\begin{enumerate}[label=\textbf{3.\arabic*}]
\item \label{dev4} Verificação dos resultados obtidos
\item \label{dev5} Novos experimentos com base nas correções
\item \label{dev6} Escrita sobre os novos experimentos
\end{enumerate}

\textbf{Etapa4: Escrita do documento e defesa do projeto}

Atividades a serem desenvolvidas:
\begin{enumerate}[label=\textbf{4.\arabic*}]
\item \label{doc1} Preparação do texto
\item \label{doc2} Preparação da apresentação
\item \label{doc3} Defesa do projeto
\end{enumerate}

\subsection{Cronograma}

\definecolor{midgray}{gray}{.5}
\begin{table}[!htbp]
	\centering
		\begin{tabular}{|c|c|c|c|c|c|c|c|c|c|c|}
		\hline
		&\multicolumn{5}{c|}{2010}&\multicolumn{5}{c|}{2010}\\
		\hline
		&MAR&ABR&MAI&JUN&JUL&AGO&SET&OUT&NOV&DEZ\\
		\hline
		\ref{b1}&\cellcolor{midgray}&&&&&&&&&\\
		\hline
		\ref{b2}&&\cellcolor{midgray}&&&&&&&&\\
		\hline	
		\ref{b3}&&\cellcolor{midgray}&&&&&&&&\\
		\hline			
		\ref{dev1}&&\cellcolor{midgray}&\cellcolor{midgray}&&&&&&&\\
		\hline	
		\ref{dev2}&&&\cellcolor{midgray}&&&&&&&\\
		\hline
		\ref{dev3}&&&\cellcolor{midgray}&\cellcolor{midgray}&&&&&&\\
		\hline	
		\ref{dev4}&&&&\cellcolor{midgray}&\cellcolor{midgray}&&&&&\\
		\hline	
		\ref{dev5}&&&\cellcolor{midgray}&\cellcolor{midgray}&\cellcolor{midgray}&&&&&\\
		\hline	
		\ref{dev6}&&&&&\cellcolor{midgray}&&&&&\\
		\hline	
		\ref{doc1}&&&&&&\cellcolor{midgray}&&&&\\
		\hline	
		\ref{doc2}&&&&&&&\cellcolor{midgray}&\cellcolor{midgray}&\cellcolor{midgray}&\\
		\hline	
		\ref{doc3}&&&&&&&&&&\cellcolor{midgray}\cellcolor{midgray}\\
		\hline	
		\end{tabular}
\end{table}
%\newpage
\section{Identificação dos demais participantes do projeto}
\label{sc:participantes}

\begin{longtable}{c c c}
% pairs: absolute number (percentage)

\toprule%
 \centering%
   \textbf{Nome}
 & \textbf{Nível}
 & \textbf{Contribuição} \\

\cmidrule[0.4pt](r{0.125em}){1-1}%
\cmidrule[0.4pt](lr{0.125em}){2-2}%
\cmidrule[0.4pt](lr{0.125em}){3-3}%
\endhead


Prof. XXX & Doutor & F.0 \\
\myrowcolour%
Prof. YYY & Mestre & F.0, F.1 \\

\bottomrule

\end{longtable}

%\newpage
%\section{Indicação de colaborações ou parcerias já estabelecidas com outros centros de pesquisa na área}
%\label{sc:colaboracoes}

%\newpage
%\section{Disponibilidade efetiva de infraestrutura e de apoio técnico para o desenvolvimento do projeto}
%\label{sc:disponibilidade}

\newpage
\section{Orientações Gerais}

Em uma monografia, pode-se justificar o tema de pesquisa, mas mais importante ainda é justificar a escolha do objetivo e da hipótese. Por exemplo, se o tema de pesquisa é “compactação de texto”, o objetivo de pesquisa é obter um algoritmo com maior grau de compactação do que os algoritmos comerciais, e a hipótese de pesquisa pode consistir em utilizar determinado modelo de rede neural para realizar essa compactação; então, a justificativa do tema deverá se concentrar em mostrar que é necessário obter algoritmos de compactação melhores. Adicionalmente, a justificativa da hipótese deverá se concentrar em apresentar evidências de que o modelo de rede neural escolhido poderá produzir resultados melhores do que os algoritmos comerciais. \cite{wazlawick2017metodologia} Cap. 6.6. 

\textbf{Leia o Livro do \cite{wazlawick2017metodologia}, mais de uma vez.}

%\section*{Referências}
\bibliographystyle{abntex2-alf}
%\bibliographystyle{ieeetr}
\bibliography{mybib.bib}

\end{document}